\documentclass[12pt]{report}
\usepackage[utf8]{inputenc}
\usepackage[english,russian]{babel} 
\usepackage{fontspec} 
\defaultfontfeatures{Ligatures={TeX},Renderer=Basic} 
\setmainfont[Ligatures={TeX,Historic}]{Times New Roman}
\usepackage[14pt]{extsizes}
\usepackage[left=2cm, right=2cm, top=1cm, bottom=2.5cm]{geometry}
\usepackage{listings}
\usepackage{titlesec, blindtext, color} % подключаем нужные пакеты
\usepackage{setspace} % полуторный интервал\\
\definecolor{gray75}{gray}{0.75} % определяем цвет
\newcommand{\hsp}{\hspace{20pt}} % длина линии в 20pt
\titleformat{\chapter}[hang]{\Huge\bfseries}{\thechapter\hsp\textcolor{gray75}\hsp}{0pt}{\Huge\bfseries}
\usepackage{graphicx}
\usepackage{color}
\definecolor{lightgray}{rgb}{.9,.9,.9}
\definecolor{darkgray}{rgb}{.4,.4,.4}
\definecolor{purple}{rgb}{0.65, 0.12, 0.82}

\lstdefinelanguage{JavaScript}{
	keywords={typeof, let, new, true, false, catch, function, for, return, null, catch, switch, var, if, in, while, do, else, case, break},
	keywordstyle=\color{blue}\bfseries,
	ndkeywords={class, export, boolean, throw, implements, import, this},
	ndkeywordstyle=\color{darkgray}\bfseries,
	identifierstyle=\color{black},
	sensitive=false,
	comment=[l]{//},
	morecomment=[s]{/*}{*/},
	commentstyle=\color{purple}\ttfamily,
	stringstyle=\color{red}\ttfamily,
	morestring=[b]',
	morestring=[b]"
}

\lstset{
	language=JavaScript,
	backgroundcolor=\color{lightgray},
	extendedchars=true,
	basicstyle=\footnotesize\ttfamily,
	showstringspaces=false,
	showspaces=false,
	numbers=left,
	numberstyle=\footnotesize,
	numbersep=9pt,
	tabsize=2,
	frame=single,
	breaklines=true,
	showtabs=false,
	captionpos=t
}
\begin{document}
	\begin{titlepage}
		\begin{table}[ht]
			\centering
			\begin{tabular}{|c|p{420pt}|} 
				\hline
				\begin{tabular}[c]{@{}c@{}} \includegraphics[scale=0.37]{EmblemBMSTU} \\\end{tabular} &
				\footnotesize\begin{tabular}[c]{@{}c@{}}\textbf{Министерство~науки~и~высшего~образования~Российской~Федерации}\\\textbf{Федеральное~государственное~бюджетное~образовательное~учреждение}\\\textbf{~высшего~образования}\\\textbf{«Московский~государственный~технический~университет}\\\textbf{имени~Н.Э.~Баумана}\\\textbf{(национальный~исследовательский~университет)»}\\\textbf{(МГТУ~им.~Н.Э.~Баумана)}\\\end{tabular}  \\
				\hline
			\end{tabular}
		\end{table}
		\noindent\rule{\textwidth}{4pt}
		\noindent\rule[14pt]{\textwidth}{1pt}
		\hfill 
		\noindent
		\makebox{ФАКУЛЬТЕТ~}%
		\makebox[\textwidth][l]{\underline{~~~~«Информатика и системы управления»~~~~~~~~~~~~~~~~~~~~~~~~~~~~~~~~~~~~~~~~~~~~}}%
		\\
		\noindent
		\makebox{КАФЕДРА~}%
		\makebox[\textwidth][l]{\underline{~~~~~~~«Программное обеспечение ЭВМ и информационные технологии»~~~~~~~~}}%
		\\
		
		
		\begin{center}
			\vspace{3cm}
			{\bf\huge Отчёт\par}
			{\bf\Large по лабораторной работе № 1\par}
			\vspace{0.5cm}
		\end{center}
		
		
		\noindent
		\makebox{\large{\bf Название:}~~~}
		\makebox[\textwidth][l]{\large\underline{~Основные элементы синтаксиса языка JavaScript~}}\\
		
		\noindent
		\makebox{\large{\bf Дисциплина:}~~~}
		\makebox[\textwidth][l]{\large\underline{~Архитектура ЭВМ~}}\\
		
		\vspace{1.5cm}
		\noindent
		\begin{tabular}{l c c c c c}
			Студент      & ~ИУ7-55Б~               & \hspace{3.5cm} & \hspace{3.5cm}                 & &  Д.О. Склифасовский \\\cline{2-2}\cline{4-4} \cline{6-6} 
			\hspace{3cm} & {\footnotesize(Группа)} &                & {\footnotesize(Подпись, дата)} & & {\footnotesize(И.О. Фамилия)}
		\end{tabular}
		
		\vspace{1cm}
		
		\noindent
		\begin{tabular}{l c c c c}
			Преподователь & \hspace{6cm}   & \hspace{3.5cm}                 & & ~~~~~~ А.Ю. Попов ~~~~~~\\\cline{3-3} \cline{5-5} 
			\hspace{3cm}  &                & {\footnotesize(Подпись, дата)} & & {\footnotesize(И.О. Фамилия)}
		\end{tabular}
		
		\begin{center}	
			\vfill
			\large \textit {Москва, 2020}
		\end{center}
		
		\thispagestyle {empty}
		\pagebreak
	\end{titlepage}
	\restoregeometry
	
	\onehalfspacing
	\tableofcontents
	
	\newpage
	\chapter*{Введение}
	\addcontentsline{toc}{chapter}{Введение}
	Цель работы: познакомиться с языком программирования JavaScript. Изучить основы ООП.\par
	\noindent В ходе лабораторной работы предстоит:
	\begin{itemize}
		\item Выполнить 2 таска, в каждом из которых по 3 задания.
	\end{itemize}
	
	\chapter{Task 1}
	\section{Задание 1}
	\noindent Создать хранилище в оперативной памяти для хранения информации о детях.\par
	\noindent Необходимо хранить информацию о ребенке: фамилия и возраст.\par
	\noindent Необходимо обеспечить уникальность фамилий детей.\par
	\noindent Реализовать функции:
	\begin{itemize}
		\item CREATE READ UPDATE DELETE для детей в хранилище
		\item Получение среднего возраста детей
		\item Получение информации о самом старшем ребенке
		\item Получение информации о детях, возраст которых входит в заданный отрезок
		\item Получение информации о детях, фамилия которых начинается с заданной буквы
		\item Получение информации о детях, фамилия которых длиннее заданного количества символов
		\item Получение информации о детях, фамилия которых начинается с гласной буквы
	\end{itemize}
	\noindent\textbf{Решение задания 1}
	\begin{lstlisting}[label=Task1.1,caption=Файл index.js]
		"use strict";
		
		function checkData(Kids, newObj){
			for (let i = 0; i < Kids.length; i++){
				if (Kids[i] === newObj){
					return 0;
				}
			}
			return 1;
		}
		
		function CREATE (Kids, newKid){
			if (!checkData(Kids, newKid)) {
				console.log("This child is already in the store.");
			}       
			else {
				Kids.push(newKid);
			}
		}
		
		function READ (Kids, name){
			let checkRes = 0;
			for (let i = 0; i < Kids.length; i++){
				if (Kids[i].surname === name){
					console.log("Age", name, ":", Kids[i].age);
					checkRes = 1;
					break;
				}
			}
			if (!checkRes){
				console.log("This child was not found.");
			}
		}
		
		function UPDATE(Kids, name, age){
			let checkRes = 0;
			for (let i = 0; i < Kids.length; i++){
				if (Kids[i].surname === name){
					Kids[i].age = age;
					checkRes = 1;
					break;
				}
			}
			if (!checkRes){
				console.log("This child was not found.");
			}
		}
		
		function DELETE(Kids, name){
			let checkRes = 0;
			let index = -1;
			for (let i = 0; i < Kids.length; i++){
				if (Kids[i].surname == name){
					checkRes = 1;
					index = i;
					break;
				}
			}
			if (checkRes){
				Kids.splice(index, 1);
			}
			else{
				console.log("This child was not found.");
			}
		}
		
		function getAverageAge(Kids){
			let sum = 0;
			for (let i = 0; i < Kids.length; i++){
				sum += Kids[i].age;
			}
			return sum / Kids.length;
		}
		
		function getInfoOnOldestKid(Kids){
			let maxAge = 0,
			index = 0;
			for (let i = 0; i < Kids.length; i++){
				if (maxAge < Kids[i].age){
					maxAge = Kids[i].age;
					index = i;
				}
			}
			console.log("Oldest child", Kids[index].surname, ", age :", Kids[index].age);
		}
		
		function getInfoOnSegment(Kids, indexFrom, indexTo){
			for (let i = indexFrom; i <= indexTo; i++){
				console.log("Name:", Kids[i].surname, ", age:", Kids[i].age);
			}
		}
		
		function getInfoByFirstLetter(Kids, letter){
			let checkRes = 0;
			for (let i = 0; i < Kids.length; i++){
				if (Kids[i].surname[0] === letter){
					checkRes = 1;
					console.log("Surname:", Kids[i].surname, ", age:", Kids[i].age);
				}
			}
			if (!checkRes){
				console.log("No one child was found.");
			}
		}
		
		function getInfoOnCountOfLetters(Kids, count){
			let checkRes = 0;
			for (let i = 0; i < Kids.length; i++){
				if (Kids[i].surname.length > count){
					checkRes = 1;
					console.log("Surname:", Kids[i].surname, ", Age:", Kids[i].age);
				}
			}
			if (!checkRes){
				console.log("No one child was found.");
			}
		}
		
		function checkInLine(first, second){
			for (let i = 0; i < second.length; i++){
				if (first === second[i]){
					return 1;
				}
			}
			return 0;
		}
		
		function getInfoByVowel(Kids){
			let vowels = "AEIOU"
			let checkRes = 0;
			for (let i = 0; i < Kids.length; i++){
				if (checkInLine(Kids[i].surname[0], vowels)){
					checkRes = 1;
					console.log("Surname:", Kids[i].surname, ", Age:", Kids[i].age);
				}
			}
			if (!checkRes){
				console.log(""No one child was found."");
			}
		}
		
		let Kids = [];
		let frstObject = {"surname" : "Naydenishev", "age" : 20};
		let scndObject = {"surname" : "Syslikov", "age" : 19};
		let thrdObject = {"surname" : "Sklifasovsky", "age" : 20};
		let frthObject = {"surname" : "Orlov", "age" : 15};
		CREATE(Kids, frstObject);
		CREATE(Kids, scndObject);
		CREATE(Kids, thrdObject);
		CREATE(Kids, frthObject);
		READ(Kids, "Orlov");
		//READ(Kids, "Daniil");
		UPDATE(Kids, "Orlov", 16);
		READ(Kids, "Orlov");
		DELETE(Kids, "Orlov");
		//console.log(Kids);
		getInfoOnOldestKid(Kids);
		getInfoOnSegment(Kids, 1, 2);
		getInfoByFirstLetter(Kids, "S");
		getInfoOnCountOfLetters(Kids, 8);
		getInfoByVowel(Kids);
	\end{lstlisting}
	
	\newpage
	\section{Задание 2}
	\noindent Создать хранилище в оперативной памяти для хранения информации о студентах.
	\noindent Необходимо хранить информацию о студенте: название группы, номер студенческого билета, оценки по программированию.
	\noindent Необходимо обеспечить уникальность номеров студенческих билетов.
	\noindent Реализовать функции:
	\begin{itemize}
		\item CREATE READ UPDATE DELETE для студентов в хранилище
		\item Получение средней оценки заданного студента
		\item Получение информации о студентах в заданной группе
		\item Получение студента, у которого наибольшее количество оценок в заданной группе
		\item Получение студента, у которого нет оценок
	\end{itemize}
	
	\noindent\textbf{Решение задания 2}
	\begin{lstlisting}[label=Task1.2,caption=Файл index.js]
		"use strict";
		const StudentsHolder = require("./StudentsHolder");
		let Students = new StudentsHolder();
		let frstStud = {"group" : 55, "number" : 18491, "marks" : [4, 4, 5]};
		let scndStud = {"group" : 55, "number" : 18480, "marks" : [2, 3, 5]};
		let thrdStud = {"group" : 53, "number" : 18301, "marks" : [2, 3, 3, 4]};
		let frthStud = {"group" : 52, "number" : 18405, "marks" : [4, 4, 4]};
		let fifthStud = {"group" : 52, "number" : 18333, "marks" : [5, 5, 5]};
		let sixthStud = {"group" : 52, "number" : 18133, "marks" : []};
		Students.CREATE(frstStud);
		Students.CREATE(scndStud);
		Students.CREATE(thrdStud);
		Students.CREATE(frthStud);
		Students.CREATE(fifthStud);
		Students.CREATE(sixthStud);
		Students.READ(18491);
		Students.READ(18480);
		Students.READ(18301);
		Students.READ(18405);
		Students.READ(18333);
		console.log("\n");
		Students.UPDATE(18491, 56, [5, 2, 4]);
		Students.READ(18491);
		console.log("\n");
		Students.DELETE(18480);
		console.log(Students.Students);
		console.log("\n");
		Students.getAverageMarks(18491);
		console.log("\n");
		Students.getInfoByGroup(52);
		console.log("\n");
		Students.getInfoByMaxMarks();
		console.log("\n");
		Students.getInfoByNoMarks();	
	\end{lstlisting}
	\begin{lstlisting}[label=Task1.2,caption=Файл StudentsHolder.js]
		module.exports = class StudentsHolder {
			constructor () {
				this.Students = [];
			}
			getNeedStudent(needNumber) {
				let index = -1;
				for (let i = 0; i < this.Students.length; i++) {
					if (this.Students[i].number === needNumber) {
						index = i;
						break;
					}
				}
				return index;
			}
			
			checkData(newObj) {
				for (let i = 0; i < this.Students.length; i++) {
					if (this.Students[i].number === newObj.number) {
						return 0;
					}
				}
				return 1;
			}
			
			CREATE(newStudent) {
				if (!this.checkData(newStudent)) {
					console.log("This student is already present in the repository.");
				}       
				else {
					this.Students.push(newStudent);
				}
			}
		
			READ(needStudent) {
				let index = this.getNeedStudent(needStudent);
				if (index === -1) {
					console.log("This student was not found.");
				}
				else {
					console.log("Group:", this.Students[index].group, "Number:", this.Students[index].number, "Marks:", this.Students[index].marks);
				}
			}
			
			UPDATE(needNumber, newGroup, newMarks) {
				let index = this.getNeedStudent(needNumber);
				if (index === -1) {
					console.log("This student was not found.");
				}
				else {
					this.Students[index].group = newGroup;
					this.Students[index].marks = newMarks;
				}
			}
			
			DELETE(needNumber) {
				let index = this.getNeedStudent(needNumber);
				if (index === -1) {
					console.log("This student was not found.");
				}
				else {
					this.Students.splice(index, 1);
				}
			}
			
			sum(arr) {
				let res = 0;
				for (let i = 0; i < arr.length; i++) {
					res += arr[i];
				}
				return res;
			}
			
			getAverageMarks(needNumber) {
				let index = this.getNeedStudent(needNumber);
				if (index === -1) {
					console.log("This student was not found.");
				}
				else {
					let sum = this.sum(this.Students[index].marks);
					if (this.Students[index].marks.length != 0) {
						console.log(sum / this.Students[index].marks.length);
					}
					else {
						console.log("This student has no grades.");
					}
				}
			}
			
			getInfoByGroup(needGroup) {
				for (let i = 0; i < this.Students.length; i++) {
					if (this.Students[i].group === needGroup) {
						this.READ(this.Students[i].number);
					}
				}
			}
			
			getInfoByMaxMarks() {
				let index = -1;
				let maxMarks = 0;
				for (let i = 0; i < this.Students.length; i++) {
					if (maxMarks < this.Students[i].marks.length) {
						maxMarks = this.Students[i].marks.length;
						index = i;
					} 
				}
				if (index != -1) {
					this.READ(this.Students[index].number);
				}
				else {
					console.log("The required student was not found");
				}
			}
			
			getInfoByNoMarks() {
				for (let i = 0; i < this.Students.length; i++) {
					if (this.Students[i].marks.length === 0) {
						this.READ(this.Students[i].number);
					}
				}
			}
		} 
	\end{lstlisting}
	
	\newpage
	\section{Задание 3}
	\noindent Создать хранилище в оперативной памяти для хранения точек.\par
	\noindent Неоходимо хранить информацию о точке: имя точки, позиция X и позиция Y.\par
	\noindent Необходимо обеспечить уникальность имен точек.\par
	\noindent Реализовать функции:\par
	\begin{itemize}
		\item CREATE READ UPDATE DELETE для точек в хранилище
		\item Получение двух точек, между которыми наибольшее расстояние
		\item Получение точек, находящихся от заданной точки на расстоянии, не превышающем заданную константу
		\item Получение точек, находящихся выше / ниже / правее / левее заданной оси координат
		\item Получение точек, входящих внутрь заданной прямоугольной зоны
	\end{itemize}
	\textbf{Решение задания 3}
	\begin{lstlisting}[label=Task1.3,caption=Файл index.js]
		"use strict";
		const Point = require("./Point");
		const PointHandler = require("./PointHandler");
		const Rectangle = require("./Rectangle");
		let frstPoint = new Point("A", 5, 5);
		let scndPoint = new Point("B", 2, 2);
		let thrdPoint = new Point("C", -1, 1);
		let frthPoint = new Point("D", 1, -1);
		let fifthPoint = new Point("E", 5, -5);
		let sixthPoint = new Point("F", 10, 10);
		let Points = new PointHandler();
		Points.CREATE(frstPoint);
		Points.CREATE(scndPoint);
		Points.CREATE(thrdPoint);
		Points.CREATE(frthPoint);
		Points.CREATE(fifthPoint);
		Points.CREATE(sixthPoint);
		Points.READ("A");
		Points.READ("B");
		Points.READ("C");
		Points.READ("D");
		Points.READ("E");
		Points.READ("F");
		console.log("\n");
		Points.UPDATE("B", -2, -2);
		Points.READ("B");
		console.log("\n");
		Points.DELETE("F");
		console.log(Points.Points);
		console.log("\n");
		Points.findMaxDistance();
		console.log("\n");
		Points.findPointsByNewPoint(0, 0, 4);
		console.log("\n");
		Points.findPointsAboutGivenAxis(0);
		Points.findPointsAboutGivenAxis(1);
		console.log("\n");
		let firstVertex = new Point("", -4, 4);
		let secondVertex = new Point("", 4, 4);
		let thirdVertex = new Point("", 4, -4);
		let fourthVertex = new Point("", -4, -4);
		let curRect = new Rectangle(firstVertex, secondVertex, thirdVertex, fourthVertex);
		Points.findPointsInrectangle(curRect);
	\end{lstlisting}
	\begin{lstlisting}[label=Task1.3,caption=Файл Point.js]
		module.exports = class Point {
			constructor(name, x, y) {
				this.name = name;
				this.x = x;
				this.y = y;
			}
		}
	\end{lstlisting}
	\begin{lstlisting}[label=Task1.3,caption=Файл PointHandler.js]
		const Point = require("./Point");
		module.exports = class PointHandler {
			constructor () {
				this.Points = [];
				this.length = this.Points.length;
			}
			
			checkData(curPoint) {
				for (let i = 0; i < this.length; i++) {
					if (curPoint.name === this.Points[i].name) {
						return 0;
					}
				}
				return 1;
			}
			getPointByName(name) {
				let index = -1;
				for (let i = 0; i < this.length; i++) {
					if (this.Points[i].name === name) {
						index = i;
						break;    
					}
				}
				return index;
			}
			CREATE(curPoint) {
				if (this.checkData(curPoint)) {
					this.Points.push(curPoint);
					this.length++;
				}
				else {
					console.log("This point already exists.");
				}
			}
			READ(name) {
				let index = this.getPointByName(name);
				if (index === -1) {
					console.log("This point already exists.");
				}
				else {
					console.log("Point: " + this.Points[index].name + ", x: " + this.Points[index].x + ", y: " + this.Points[index].y + ".");
				}
			}
			UPDATE(name, x, y) {
				let index = this.getPointByName(name);
				if (index === -1) {
					console.log("This point already exists.");
				}
				else {
					this.Points[index].x = x;
					this.Points[index].y = y; 
				}
			}
			DELETE(name) {
				let index = this.getPointByName(name);
				if (index === -1) {
					console.log("This point already exists.");
				}
				else {
					this.Points.splice(index, 1);
					this.length--;
				}
			}
			getDistance(frstPoint, scndPoint) {
				return Math.sqrt(Math.pow(scndPoint.x - frstPoint.x, 2) + Math.pow(scndPoint.y - frstPoint.y, 2))
			}
			findMaxDistance() {
				let frstPoint = this.Points[0],
				scndPoint = this.Points[1];
				let maxDistance = this.getDistance(frstPoint, scndPoint);
				for (let i = 0; i < this.length - 1; i++) {
					for (let j = i + 1; j < this.length; j++) {
						let curDistance = this.getDistance(this.Points[i], this.Points[j]) 
						if (curDistance > maxDistance) {
							maxDistance = curDistance;
							frstPoint = this.Points[i];
							scndPoint = this.Points[j];
						}
					}
				}
				console.log("Maximum distance = " + maxDistance + " between points:")
				this.READ(frstPoint.name);
				this.READ(scndPoint.name);
			}
			findPointsByNewPoint(x, y, constant) {
				for (let i = 0; i < this.length; i++) {
					if (this.getDistance(new Point("CheckDistance", x, y), this.Points[i]) <= constant) {
						this.READ(this.Points[i].name);
					}
				}
			}
			// 0 - x, 1 - y
			findAboutX() {
				console.log("Above Y:");
				for (let i = 0; i < this.length; i++) {
					if (this.Points[i].y > 0) {
						this.READ(this.Points[i].name);
					}
				}
				console.log("Below Y:");
				for (let i = 0; i < this.length; i++) {
					if (this.Points[i].y < 0) {
						this.READ(this.Points[i].name);
					}
				}
			}
			findAboutY() {
				console.log("To the right X:");
				for (let i = 0; i < this.length; i++) {
					if (this.Points[i].x > 0) {
						this.READ(this.Points[i].name);
					}
				}
				console.log("To the left X:");
				for (let i = 0; i < this.length; i++) {
					if (this.Points[i].x < 0) {
						this.READ(this.Points[i].name);
					}
				}
			}
			findPointsAboutGivenAxis(axis) {
				if (!axis) {
					this.findAboutX();
				}
				else {
					this.findAboutY();
				}
			}	
			findMaxCoords(curRect) {
				let maxX = curRect.curRectangle[0].x, 
				maxY = curRect.curRectangle[0].y;
				for (let i = 0; i < 4; i++) {
					if (maxX < curRect.curRectangle[i].x) {
						maxX = curRect.curRectangle[i].x;
					}
					if (maxY < curRect.curRectangle[i].y) {
						maxY = curRect.curRectangle[i].y;
					}
				}
				return new Point("", maxX, maxY);
			}
			findMinCoords(curRect) {
				let minX = curRect.curRectangle[0].x, 
				minY = curRect.curRectangle[0].y;
				for (let i = 0; i < 4; i++) {
					if (minX > curRect.curRectangle[i].x) {
						minX = curRect.curRectangle[i].x;
					}
					if (minY > curRect.curRectangle[i].y) {
						minY = curRect.curRectangle[i].y;
					}
				}
				return new Point("", minX, minY);
			}
			IsInRect(curRect, curPoint) {
				let maxPoint = this.findMaxCoords(curRect);
				let minPoint = this.findMinCoords(curRect);
				if (curPoint.x < maxPoint.x && curPoint.x > minPoint.x && 
				curPoint.y < maxPoint.y && curPoint.y > minPoint.y) {
					return 1;
				}
				return 0;
			}
			findPointsInrectangle(curRect) {
				for (let i = 0; i < this.length; i++) {
					if (this.IsInRect(curRect, this.Points[i])) {
						this.READ(this.Points[i].name);
					}
				}
			}
		}
	\end{lstlisting}
	\begin{lstlisting}[label=Task1.3,caption=Файл Rectangle.js]
		const Point = require("./Point")
		module.exports = class Rectangle {
			constructor(firstPoint, secondPoint, thirdPoint, fourthPoint) {
				this.curRectangle = [];
				this.curRectangle.push(firstPoint);
				this.curRectangle.push(secondPoint);
				this.curRectangle.push(thirdPoint);
				this.curRectangle.push(fourthPoint);
			}
		}
	\end{lstlisting}

	\chapter{Task 2}
	\section{Задание 1}
	\noindent Создать класс Точка.\par
	\noindent Добавить классу точка Точка метод инициализации полей и метод вывода полей на экран\par
	\noindent Создать класс Отрезок.\par
	\noindent У класса Отрезок должны быть поля, являющиеся экземплярами класса Точка.\par
	\noindent Добавить классу Отрезок метод инициализации полей, метод вывода информации о полях на экран, а так же метод получения длины отрезка.\par
	\noindent\textbf{Решение задания 1}
	\begin{lstlisting}[label=Task2.1,caption=Файл index.js]
		"use strict";
		const Point = require("./Point");
		const Section = require("./Section");
		let checkSection = new Section();
		checkSection.init(0, 0, 4, 0);
		checkSection.printSection();
		let length = checkSection.findLength();
		console.log(length);
	\end{lstlisting}
	\begin{lstlisting}[label=Task2.1,caption=Файл Point.js]
		"use strict";
		module.exports = class Point {
			constructor() {
				this.x = 0;
				this.y = 0;
			}
			
			init(x, y) {
				this.x = x;
				this.y = y;
			}
			
			printPoint() {
				console.log("Point: " + this.x + ";" + this.y);
			}
		}
	\end{lstlisting}
	\begin{lstlisting}[label=Task2.1,caption=Файл Section.js]
		"use strict";
		const Point = require("./Point");
		module.exports = class Section {
			constructor() {
				this.firstPoint = new Point();
				this.secondPoint = new Point();
			}
			
			init(x1, y1, x2, y2) {
				this.firstPoint.init(x1, y1);
				this.secondPoint.init(x2, y2);
			}
			
			printSection() {
				console.log("Section: ");
				this.firstPoint.printPoint();
				this.secondPoint.printPoint();
			}
			
			findLength() {
				return Math.sqrt(Math.pow(this.secondPoint.x - this.firstPoint.x , 2) + Math.pow(this.secondPoint.y - this.firstPoint.y , 2));
			}
		}
	\end{lstlisting}
	
	\newpage
	\section{Задание 2}
	\noindent Создать класс Треугольник.\par
	\noindent Класс Треугольник должен иметь поля, хранящие длины сторон треугольника.\par
	\noindent Реализовать следующие методы:\par
	\begin{itemize}
		\item Метод инициализации полей
		\item Метод проверки возможности существования треугольника с такими сторонами
		\item Метод получения периметра треугольника
		\item Метод получения площади треугольника
		\item Метод для проверки факта: является ли треугольник прямоугольным
	\end{itemize}
	\noindent\textbf{Решение задания 2}
	\begin{lstlisting}[label=Task2.2,caption=Файл index.js]
		"use strict";
		const Triangle = require("./Tiangle");
		let checkTriangle = new Triangle();
		checkTriangle.init(5, 5, 6);
		console.log(checkTriangle.isExist());
		console.log(checkTriangle.getPerimeter());
		console.log(checkTriangle.getArea());
		console.log(checkTriangle.isRectangular());
		console.log("\n");
		let checkNewTriangle = new Triangle();
		checkNewTriangle.init(1, 1, 4);
		console.log(checkNewTriangle.isExist());
		console.log("\n");
		checkNewTriangle.init(3, 4, 5);
		console.log(checkNewTriangle.isExist());
		console.log(checkNewTriangle.isRectangular());
	\end{lstlisting}
	\begin{lstlisting}[label=Task2.2,caption=Файл Triangle.js]
		"use strict";
		
		module.exports = class Triangle {
			constructor() {
				this.lenA = 0;
				this.lenB = 0;
				this.lenC = 0;
			}
			
			init(len1, len2, len3) {
				this.lenA = len1;
				this.lenB = len2;
				this.lenC = len3;
			}
			
			isExist() {
				if (this.lenA + this.lenB > this.lenC && 
				this.lenA + this.lenC > this.lenB &&
				this.lenB + this.lenC > this.lenA) {
					return 1;
				}
				return 0;
			}
			
			getPerimeter() {
				if (this.isExist()) {
					return this.lenA + this.lenB + this.lenC;
				}
				else {
					console.log("There is no triangle.");
					return -1;
				}
			}
			
			getArea() {
				if (this.isExist()) {
					let perimeter = this.getPerimeter();
					return Math.sqrt(perimeter * (perimeter - this.lenA) * (perimeter - this.lenB) * (perimeter - this.lenC));
				}
				else {
					console.log("There is no triangle.");
					return -1;
				}
			}
			
			checkSide(firstLen, secondLen, thirdLen) {
				if (Math.pow(firstLen, 2) + Math.pow(secondLen, 2) === Math.pow(thirdLen, 2)) {
					return 1;
				}
				return 0;
			}
			
			isRectangular() {
				if (this.isExist()) {
					if (this.checkSide(this.lenA, this.lenB, this.lenC) || this.checkSide(this.lenA, this.lenC, this.lenB) || this.checkSide(this.lenB, this.lenC, this.lenA)) {
						return 1;
					}
					else {
						return 0;
					}
				}
				else {
					console.log("There is no triangle.");
					return -1;
				}
			}
		}
	\end{lstlisting}
	
	\newpage
	\section{Задание 3}
	\noindent Реализовать программу, в которой происходят следующие действия:\par
	\noindent Происходит вывод целых чисел от 1 до 10 с задержками в 2 секунды.\par
	\noindent После этого происходит вывод от 11 до 20 с задержками в 1 секунду.\par
	\noindent Потом опять происходит вывод чисел от 1 до 10 с задержками в 2 секунды.\par
	\noindent После этого происходит вывод от 11 до 20 с задержками в 1 секунду.\par
	\noindent Это должно происходить циклически.\par
	\noindent\textbf{Решение задания 2}	
	\begin{lstlisting}[label=Task2.3,caption=Файл index.js]
		"use strict";
		
		function checkInterval(startNumber, endNumber, intervalTime, i) {
			let interval = setInterval(() => {
				console.log(startNumber);
				startNumber++;
				if (startNumber == endNumber + 1) {
					clearInterval(interval);
					i++;
					let start = i % 2 * 10 + 1;
					let end = start + 9;
					let intTime = 1000;
					if (i % 2 == 0) {
						intTime = 2000;
					}
					checkInterval(start, end, intTime, i);
				}
			}, intervalTime);
		}

		checkInterval(1, 10, 2000, 0);

	\end{lstlisting}
	
	\restoregeometry
	\newpage
	\chapter*{Заключение}
	\addcontentsline{toc}{chapter}{Заключение}
	В ходе выполнения данной лабораторной работой я познакомился с языком программирования JavaScript и изучил основы ООП.
\end{document}